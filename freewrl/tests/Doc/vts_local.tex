%% $Id: vts_local.tex,v 1.4 2001/07/26 17:56:50 ayla Exp $
%% break this up into multiple files!!!

\documentclass[12pt,letterpaper]{article}
\usepackage{times}
\usepackage{pifont}
\usepackage{longtable}

%% Margins
\setlength{\topmargin}{0mm}
\setlength{\voffset}{0mm}
\setlength{\hoffset}{0mm}
\setlength{\oddsidemargin}{0mm}
\setlength{\marginparsep}{0mm}
\setlength{\marginparwidth}{0mm}
\setlength{\marginparpush}{0mm}
\setlength{\textwidth}{158mm}
\setlength{\textheight}{206mm}

%% Counters
\newcounter{testCaseCtr}

%% New Commands
\newcommand{\testCase}{\arabic{testCaseCtr}.~ \stepcounter{testCaseCtr}}
\newcommand{\resetTestCase}{\setcounter{testCaseCtr}{1}}

%% Titles for test cases.
%% This saves some typing, although the names are a bit unreadable,
%% and letters need to be used instead of numbers.
%% This should be moved to another file.

%% Appearance: Appearance test cases
\newcommand{\AppA}{Material Field}
\newcommand{\AppB}{ImageTexture Applied To Texture Field}
\newcommand{\AppC}{PixelTexture Applied To Texture Field}
\newcommand{\AppDa}{Default Color, Opacity For Unlit, Uncolored,~}
\newcommand{\AppDb}{Untextured Geometry}
\newcommand{\AppE}{PointSet, IndexedLineSet With Default Color}
\newcommand{\AppF}{Default Color For Unlit, Untextured Geometry}
\newcommand{\AppGa}{ImageTexture Applied To Unlit,~}
\newcommand{\AppGb}{Uncolored Geometry}
\newcommand{\AppHa}{Color, Greyscale ImageTexture Applied~}
\newcommand{\AppHb}{To Unlit IndexedFaceSet}
\newcommand{\AppIa}{Intensity, Greyscale ImageTexture Applied~}
\newcommand{\AppIb}{To Unlit IndexedFaceSet}
\newcommand{\AppJa}{Color Precedence Of Color ImageTexture Over~}
\newcommand{\AppJb}{Color Node}
\newcommand{\AppKa}{Color Transparent ImageTexture Applied To~}
\newcommand{\AppKb}{Unlit Geometry}
\newcommand{\AppLa}{Greyscale Transparent ImageTexture~}
\newcommand{\AppLb}{Applied To Unlit Geometry}

%% Appearance: FontStyle test cases
\newcommand{\FSA}{FontStyle With Default Values}
\newcommand{\FSB}{FontStyle With 3 Family Types}
\newcommand{\FSC}{FontStyle With Next Supported Family}
\newcommand{\FSD}{FontStyle With Unsupported Family}
\newcommand{\FSE}{FontStyle With All Families And Supported Styles}
\newcommand{\FSF}{FontStyle With 3 Size, Spacing Values}
\newcommand{\FSG}{FontStyle With Text Orientation, Spacing}

%% Appearance: ImageTexture test cases
\newcommand{\ITA}{JPEG RGB Texture Mapping To Primitive Geometry}
\newcommand{\ITB}{PNG RGB, Palette Mapping To Primitive Geometry}
\newcommand{\ITC}{JPEG Greyscale Texture Mapping To Primitive Geometry}
\newcommand{\ITD}{PNG Greyscale Texture Mapping To Primitive Geometry}
\newcommand{\ITE}{256 X 256 Pixel Minimum File Requirement For JPEG Files}
\newcommand{\ITF}{256 X 256 Pixel Minimum File Requirement For PNG Files}
\newcommand{\ITG}{Default IndexedFaceSet Geometry Mapping For JPEG Files}
\newcommand{\ITHa}{Default IndexedFaceSet Geometry Mapping For Bounding Box~}
\newcommand{\ITHb}{With Equal Length Sides, JPEG Files}
\newcommand{\ITIa}{Default IndexedFaceSet Geometry Mapping For Bounding Box~}
\newcommand{\ITIb}{With Equal Length Smaller Sides, JPEG Files}
\newcommand{\ITJa}{Default IndexedFaceSet Geometry Mapping For Bounding Box}
\newcommand{\ITJb}{With Equal Length Longer Sides, JPEG Files}
\newcommand{\ITK}{Default JPEG Mapping To ElevationGrid Geometry}
\newcommand{\ITL}{Default JPEG Mapping To Extrusion Geometry}
\newcommand{\ITM}{Default JPEG Mapping To Text Geometry}
\newcommand{\ITN}{Default IndexedFaceSet Geometry Mapping For PNG Files}
\newcommand{\ITOa}{Default IndexedFaceSet Geometry Mapping For Bounding Box~}
\newcommand{\ITOb}{With Equal Length Sides, PNG Files}
\newcommand{\ITPa}{Default IndexedFaceSet Geometry Mapping For Bounding Box~}
\newcommand{\ITPb}{With Equal Length Smaller Sides, PNG Files}
\newcommand{\ITQa}{Default IndexedFaceSet Geometry Mapping For Bounding Box}
\newcommand{\ITQb}{With Equal Length Longer Sides, PNG Files}
\newcommand{\ITR}{Default PNG Mapping To ElevationGrid Geometry}
\newcommand{\ITS}{Default PNG Mapping To Extrusion Geometry}
\newcommand{\ITT}{Default PNG Mapping To Text Geometry}
\newcommand{\ITUa}{PNG Greyscale, Alpha-Opacity, Binary Transparency~}
\newcommand{\ITUb}{Texture Mapping To Primitive Geometry}
\newcommand{\ITVa}{PNG Greyscale, Alpha-Opacity, Binary Transparency~}
\newcommand{\ITVb}{Texture Mapping To Complex Geometry}
\newcommand{\ITWa}{PNG RGB, Alpha-Opacity, Binary Transparency~}
\newcommand{\ITWb}{Texture Mapping To Primitive Geometry}
\newcommand{\ITXa}{PNG RGB, Alpha-Opacity, Binary Transparency~}
\newcommand{\ITXb}{Texture Mapping To Complex Geometry}
\newcommand{\ITY}{JPEG Greyscale, DiffuseColor Combination}
\newcommand{\ITZ}{JPEG Greyscale, Color Node Combination}
\newcommand{\ITAA}{PNG Greyscale, DiffuseColor Combination}
\newcommand{\ITAB}{PNG Greyscale, Color Node Combination}
\newcommand{\ITAC}{Unclamped Horizontal, Vertical ImageTexture JPEG Mapping}
\newcommand{\ITAD}{Vertical Clamping Of ImageTexture JPEG Mapping}
\newcommand{\ITAE}{Horizontal Clamping Of ImageTexture JPEG Mapping}
\newcommand{\ITAF}{Unclamped Horizontal, Vertical ImageTexture PNG Mapping}
\newcommand{\ITAG}{Vertical Clamping Of ImageTexture PNG Mapping}
\newcommand{\ITAH}{Horizontal Clamping Of ImageTexture PNG Mapping}

%% Appearance: Material test cases
\newcommand{\MatA}{Default Material Node Values For Primitive Geometry}
\newcommand{\MatB}{Default Material Node Values For Complex Geometry}
\newcommand{\MatC}{5 DiffuseColor Values For Primitive Geometries}
\newcommand{\MatD}{5 Shininess Values For Primitive Geometries}
\newcommand{\MatE}{5 DiffuseColor Values For Complex Geometries}
\newcommand{\MatF}{5 Shininess Values For Complex Geometries}
\newcommand{\MatGa}{Preference Of Texture Color Over Material DiffuseColor~}
\newcommand{\MatGb}{For Primitive Geometry}
\newcommand{\MatHa}{Preference Of Texture Color Over Material DiffuseColor~}
\newcommand{\MatHb}{For Complex Geometry}
\newcommand{\MatIa}{Preference Of Texture Color Over Geometry Color~}
\newcommand{\MatIb}{For Complex Geometry}
\newcommand{\MatJa}{Preference Of DiffuseColor Blending With Greyscale Texture~}
\newcommand{\MatJb}{When Geometry Color Is NULL}
\newcommand{\MatKa}{Preference Of Geometry Color Blending With Greyscale Texture~}
\newcommand{\MatKb}{Over Material Node DiffuseColor When Both Are Present}
\newcommand{\MatL}{Preference Of Geometry Color Over Material Node DiffuseColor}
\newcommand{\MatMa}{Preference Of DiffuseColor Geometry Coloring When~}
\newcommand{\MatMb}{Geometry Color Node is NULL}
\newcommand{\MatNa}{Preference Of Geometry Color Over Material EmissiveColor~}
\newcommand{\MatNb}{For PointSet Geometry}
\newcommand{\MatOa}{Preference Of EmissiveColor When Geometry Color Is NULL~}
\newcommand{\MatOb}{For PointSet Geometry}
\newcommand{\MatPa}{Preference Of Geometry Color Over Material EmissiveColor~}
\newcommand{\MatPb}{For IndexedLineSet Geometry}
\newcommand{\MatQa}{Preference Of EmissiveColor When Geometry Color Is NULL~}
\newcommand{\MatQb}{For IndexedLineSet Geometry}
\newcommand{\MatRa}{Preference Of Color Texture Alpha-Opacity Over~}
\newcommand{\MatRb}{Material Node Transparency Field}
\newcommand{\MatSa}{Preference Of Greyscale Texture Alpha-Opacity Over~}
\newcommand{\MatSb}{Material Node Transparency Field}
\newcommand{\MatTa}{Preference Of Material Transparency When Color Texture~}
\newcommand{\MatTb}{Has No Alpha-Opacity Value}
\newcommand{\MatUa}{Preference Of Material Transparency When Greyscale Texture~}
\newcommand{\MatUb}{Has No Alpha-Opacity Value}
\newcommand{\MatV}{5 AmbientIntensity Values For Primitive Geometries}
\newcommand{\MatW}{5 EmissiveColor Values For Primitive Geometries}
\newcommand{\MatX}{5 SpecularColor Values For Primitive Geometries}
\newcommand{\MatY}{5 AmbientIntensity Values For Complex Geometries}
\newcommand{\MatZ}{5 EmissiveColor Values For Complex Geometries}
\newcommand{\MatAA}{5 SpecularColor Values For Complex Geometries}
\newcommand{\MatAB}{5 Transparency Values For Primitive Geometries}
\newcommand{\MatAC}{5 Transparency Values For Complex Geometries}

%% Appearance: TextureTransform test cases
\newcommand{\TTA}{Default TextureTransform Node To Primitive Geometries}
\newcommand{\TTBa}{Default TextureTransform Mapping Of ImageTexture}
\newcommand{\TTBb}{For Extrusion Geometry}
\newcommand{\TTCa}{Default TextureTransform Mapping Of ImageTexture}
\newcommand{\TTCb}{For ElevationGrid Geometry}
\newcommand{\TTDa}{Default TextureTransform Mapping Of ImageTexture}
\newcommand{\TTDb}{For IndexedFaceSet Geometry}
\newcommand{\TTEa}{Scale Field Applied To ImageTexture Mapped}
\newcommand{\TTEb}{To Primitive Geometry}
\newcommand{\TTFa}{Scale Field Applied To ImageTexture Mapped}
\newcommand{\TTFb}{To Complex Geometry}
\newcommand{\TTGa}{Rotation Field Applied To ImageTexture Mapped}
\newcommand{\TTGb}{To Primitive Geometry}
\newcommand{\TTHa}{Rotation Field Applied To ImageTexture Mapped}
\newcommand{\TTHb}{To Complex Geometry}

%% Bindable Nodes: Background test cases
\newcommand{\BkgA}{Simple (Single) Binded Background Sending TRUE Events}
\newcommand{\BkgB}{Simple (Single) Non Binded Background Sending FALSE Events}
\newcommand{\BkgCa}{Sending FALSE Event To Background At Stack Top Will Bind}
\newcommand{\BkgCb}{Next Background In Stack}
\newcommand{\BkgDa}{Sending TRUE Event To Background Not At Stack Top Should}
\newcommand{\BkgDb}{Force Active Background Out Of Stack Top}
\newcommand{\BkgE}{Multiple Backgrounds Being Activated/Deactivated Dynamically}
\newcommand{\BkgFa}{When Last Angle Is $ < \pi$ Last Should Clamp From Last Angle}
\newcommand{\BkgFb}{To Nadir}
\newcommand{\BkgGa}{When Last Angle Is $ < \frac{\pi}{2}$ Then Last Angle/Equator}
\newcommand{\BkgGb}{Region Should Be Invisible}
\newcommand{\BkgH}{}
\newcommand{\BkgI}{}
\newcommand{\BkgJ}{}
\newcommand{\BkgK}{}
\newcommand{\BkgL}{}
\newcommand{\BkgM}{}
\newcommand{\BkgN}{}
\newcommand{\BkgO}{}
\newcommand{\BkgP}{}
\newcommand{\BkgQ}{}
\newcommand{\BkgR}{}
\newcommand{\BkgS}{}
\newcommand{\BkgT}{}
\newcommand{\BkgU}{}
\newcommand{\BkgV}{}
\newcommand{\BkgW}{}

%% Bindable Nodes: Fog test cases
\newcommand{\FogA}{Default Browser Behavior When No Fog Nodes Are Present}

%% Bindable Nodes: NavigationInfo test cases

%% Bindable Nodes: Viewpoint test cases

%% Special Groups: Inline test cases
\newcommand{\SpGrA}{Single Url Item}
\newcommand{\SpGrB}{Inline Node Referencing Another Inline Node}
\newcommand{\SpGrC}{Inline Node With Invalid Url Reference}
\newcommand{\SpGrD}{Test Absolute Urls}
\newcommand{\SpGrE}{Test Relative Urls}
%% Beyond minimum conformance requirements: Bounding boxes
\newcommand{\SpGrF}{Bounding Box Size Equal To Children's Size}

%% Special Groups: LOD test cases
%% Special Groups: Switch test cases

\begin{document}
\part*{NIST VRML Test Suite - Local FreeWRL Compliance Testing}

Testing the FreeWRL VRML97 browser locally on think.dgrc.crc.ca.
FreeWRL has been configured as a Netscape plug-in.
The testing was done with Netscape Communicator 4.61.
The web server used is Apache 1.3.19.
Results are grouped by Node Group.

\section{Appearance Test Cases}
\resetTestCase

\begin{center}
\setlongtables
\begin{longtable}{|l|l|l|}
%% Table head and foot rows
\hline
\textbf{Appearance} & \textbf{Test Case} & \textbf{Result} \\\hline\hline
\endhead
%%\cline{2-3}
 & \multicolumn{2}{|r|}{\textsl{continued on the next page\ldots}} \\\hline
\endfoot\hline
\endlastfoot

Appearance & \testCase \AppA & Passed \\\cline{2-3}
 & \testCase \AppB & Passed \\\cline{2-3}
 & \testCase \AppC & Undet. \\\cline{2-3}
 & \testCase \AppDa & Failed \\
 & \AppDb & \\\cline{2-3}
 & \testCase \AppE & Undet. \\\cline{2-3}
 & \testCase \AppF & Failed \\\cline{2-3}
 & \testCase \AppGa & Failed \\
 & \AppGb & \\\cline{2-3}
 & \testCase \AppHa & Failed \\
 & \AppHb & \\\cline{2-3}
 & \testCase \AppIa & Failed \\
 & \AppIb & \\\cline{2-3}
 & \testCase \AppJa & Failed \\
 & \AppJb & \\\cline{2-3}
 & \testCase \AppKa & Failed \\
 & \AppKb & \\\cline{2-3}
 & \testCase \AppLa & Failed \\
 & \AppLb & \\\hline
\resetTestCase
FontStyle & \testCase \FSA & Failed \\\cline{2-3}
 & \testCase \FSB & Failed \\\cline{2-3}
 & \testCase \FSC & Failed \\\cline{2-3}
 & \testCase \FSD & Failed \\\cline{2-3}
 & \testCase \FSE & Failed \\\cline{2-3}
 & \testCase \FSF & Failed \\\cline{2-3}
 & \testCase \FSG & Failed \\\hline
\resetTestCase
ImageTexture & \testCase \ITA & Passed \\\cline{2-3}
 & \testCase \ITB & Passed \\\cline{2-3}
 & \testCase \ITC & Passed \\\cline{2-3}
 & \testCase \ITD & Passed \\\cline{2-3}
 & \testCase \ITE & Passed \\\cline{2-3}
 & \testCase \ITF & Passed \\\cline{2-3}
 & \testCase \ITG & Failed \\\cline{2-3}
 & \testCase \ITHa & Failed \\
 & \ITHb & \\\cline{2-3}
 & \testCase \ITIa & Failed \\
 & \ITIb & \\\cline{2-3}
 & \testCase \ITJa & Failed \\
 & \ITJb & \\\cline{2-3}
 & \testCase \ITK & Passed \\\cline{2-3}
 & \testCase \ITL & Failed \\\cline{2-3}
 & \testCase \ITM & Failed \\\cline{2-3}
 & \testCase \ITN & Failed \\\cline{2-3}
 & \testCase \ITOa & Failed \\
 & \ITOb & \\\cline{2-3}
 & \testCase \ITPa & Failed \\
 & \ITPb & \\\cline{2-3}
 & \testCase \ITQa & Failed \\
 & \ITQb & \\\cline{2-3}
 & \testCase \ITR & Passed \\\cline{2-3}
 & \testCase \ITS & Failed \\\cline{2-3}
 & \testCase \ITT & Failed \\\cline{2-3}
 & \testCase \ITUa & Failed \\
 & \ITUb & \\\cline{2-3}
 & \testCase \ITVa & Failed \\
 & \ITVb & \\\cline{2-3}
 & \testCase \ITWa & Passed \\\cline{2-3}
 & \ITWb & \\\cline{2-3}
 & \testCase \ITXa & Failed \\
 & \ITXb & \\\cline{2-3}
 & \testCase \ITY & Passed \\\cline{2-3}
 & \testCase \ITZ & Failed \\\cline{2-3}
 & \testCase \ITAA & Passed \\\cline{2-3}
 & \testCase \ITAB & Failed \\\cline{2-3}
 & \testCase \ITAC & Passed \\\cline{2-3}
 & \testCase \ITAD & Failed \\\cline{2-3}
 & \testCase \ITAE & Failed \\\cline{2-3}
 & \testCase \ITAF & Passed \\\cline{2-3}
 & \testCase \ITAG & Failed \\\cline{2-3}
 & \testCase \ITAH & Failed \\\hline
\resetTestCase
Material & \testCase \MatA & Passed \\\cline{2-3}
 & \testCase \MatB & Passed \\\cline{2-3}
 & \testCase \MatC & Passed \\\cline{2-3}
 & \testCase \MatD & Passed \\\cline{2-3}
 & \testCase \MatE & Passed \\\cline{2-3}
 & \testCase \MatF & Passed \\\cline{2-3}
 & \testCase \MatGa & Failed \\
 & \MatGb & \\\cline{2-3}
 & \testCase \MatHa & Failed \\
 & \MatHb & \\\cline{2-3}
 & \testCase \MatIa & Failed \\
 & \MatIb & \\\cline{2-3}
 & \testCase \MatJa & Failed \\
 & \MatJb & \\\cline{2-3}
 & \testCase \MatKa & Passed \\
 & \MatKb & \\\cline{2-3}
 & \testCase \MatL & Passed \\\cline{2-3}
 & \testCase \MatMa & Passed \\
 & \MatMb & \\\cline{2-3}
 & \testCase \MatNa & Failed \\
 & \MatNb & \\\cline{2-3}
 & \testCase \MatOa & \\
 & \MatOb & \\\cline{2-3}
 & \testCase \MatPa & \\
 & \MatPb & \\\cline{2-3}
 & \testCase \MatQa & \\
 & \MatQb & \\\cline{2-3}
 & \testCase \MatRa & \\
 & \MatRb & \\\cline{2-3}
 & \testCase \MatSa & \\
 & \MatSb & \\\cline{2-3}
 & \testCase \MatTa & \\
 & \MatTb & \\\cline{2-3}
 & \testCase \MatUa & \\
 & \MatUb & \\\cline{2-3}
 & \testCase \MatV & \\\cline{2-3}
 & \testCase \MatW & \\\cline{2-3}
 & \testCase \MatX & \\\cline{2-3}
 & \testCase \MatY & \\\cline{2-3}
 & \testCase \MatZ & \\\cline{2-3}
 & \testCase \MatAA & \\\cline{2-3}
 & \testCase \MatAB & \\\cline{2-3}
 & \testCase \MatAC & \\\hline
\resetTestCase
MovieTexture & \multicolumn{2}{|c|}{\textbf{MovieTexture not supported.}}\\\hline
\resetTestCase
PixelTexture & & \\\hline
\resetTestCase
Texture- & \testCase \TTA & Passed \\\cline{2-3}
Transform & \testCase \TTBa & Failed \\
 & \TTBb & \\\cline{2-3}
 & \testCase \TTCa & Passed \\
 & \TTCb & \\\cline{2-3}
 & \testCase \TTDa & Passed \\
 & \TTDb & \\\cline{2-3}
 & \testCase \TTEa & Failed \\
 & \TTEb & \\\cline{2-3}
 & \testCase \TTFa & Failed \\
 & \TTFb & \\\cline{2-3}
 & \testCase \TTGa & Failed \\
 & \TTGb & \\\cline{2-3}
 & \testCase \TTHa & Failed \\
 & \TTHb & \\\cline{2-3}
\end{longtable}
\end{center}

\subsection{Appearance Test Cases}

%%\subsubsection{Material Field}
\subsubsection{\AppA}
The VRML file is material.wrl.

\subsubsection{\AppB}
\label{sec:imagetexture}
The VRML file is imagetexture.wrl, and the JPEG ImageTexture is vts.jpg.

\subsubsection{\AppC}
The VRML file is pixeltexture.wrl.\newline
Netscape either stalls or runs out of memory.
Look into this later.

\subsubsection{\AppDa\AppDb}
\label{sec:unlit}
The VRML file is unlit.wrl.

\subsubsection{\AppE}
The VRML file is unlit\_pointline.wrl.\newline
The IndexedLineSet and PointSet are visible for a moment, then the scene becomes completely white. 
The same behavior is observed in FreeWRL as a stand-alone application.\newline
The description of the test case does state that since PointSet and IndexedLineSet do not support
lighting, test results are undetermined.

\subsubsection{\AppF}
The VRML file is unlit\_color.wrl.\newline
FreeWRL crashes as both plug-in and stand-alone application:
\begin{verbatim}
Done the shape
Gen elevgrid 800 21 21
Done the shape
Done the shape
Not same number of colors and points a
 /usr/lib/perl5/site_perl/i386-linux/VRML/GLBackEnd.pm line 708.
VRML::GLBackEnd::render('VRML::GLBackEnd=HASH(0x815a420)')
called at /usr/lib/perl5/site_perl/i386-linux/VRML/GLBackEnd.pm
line 220
VRML::GLBackEnd::update_scene('VRML::GLBackEnd=HASH(0x815a420)',
989519064.2)
called at /usr/lib/perl5/site_perl/i386-linux/VRML/Browser.pm
line 204
VRML::Browser::tick('VRML::Browser=HASH(0x86b97b4)')
called at /usr/lib/perl5/site_perl/i386-linux/VRML/Browser.pm
line 127
VRML::Browser::eventloop('VRML::Browser=HASH(0x86b97b4)')
called at /usr/bin/freewrl line 551
Died at /usr/bin/freewrl line 391.
\end{verbatim}

\subsubsection{\AppGa\AppGb}
\label{sec:unlit-texture-nocolor}
The VRML file is unlit\_texture\_nocolor.wrl, and the JPEG ImageTexture is vts.jpg.\newline
The ImageTexture does not get applied to the IndexedFaceSet correctly.

\subsubsection{\AppHa\AppHb}
The VRML file is unlit\_greyscale\_color\_blend.wrl, and the JPEG ImageTexture
is greyscalecolor.jpg.\newline
The results are similar to section~\ref{sec:unlit-texture-nocolor}.

\subsubsection{\AppIa\AppIb}
The VRML file is unlit\_greyscale\_intensity.wrl, and the JPEG ImageTexture
is greyscalecolor.jpg.\newline
The results are similar to section~\ref{sec:unlit-texture-nocolor}.
As is section~\ref{sec:unlit}, the background appears stark white, not grey.

\subsubsection{\AppJa\AppJb}
The VRML file is unlit\_texture\_color.wrl, and the JPEG ImageTexture
is greyscalecolor.jpg.\newline
The geometry Color Node is preferred over texture color and, as
in~\ref{sec:unlit-texture-nocolor}, the ImageTexture does not get
applied to the IndexedFaceSet correctly.

\subsubsection{\AppKa\AppKb}
\label{sec:unlit-rgb-texture}
The VRML file is unlit\_rgb\_texture\_alpha.wrl, and the ImageTexture
is rgb\_alpha.png.\newline
The ImageTexture does not get applied to the IndexedFaceSet correctly
and does not get applied to the Extrusion at all.

\subsubsection{\AppLa\AppLb}
The VRML file is unlit\_greyscale\_texture\_alpha.wrl, and the ImageTexture
is greyscale\_alpha.png.\newline
The results are similar to section~\ref{sec:unlit-rgb-texture}.

\subsection{FontStyle Test Cases}

\subsubsection{\FSA}
\label{sec:fs-default}
The VRML file is Appearance/FontStyle/default.wrl.
The FreeWRL generated text does not resemble the expected result when FreeWRL is both 
a stand-alone application and plug-in.

\subsubsection{\FSB}
\label{sec:3-families}
The VRML file is family.wrl.\newline
The results are similar to section~\ref{sec:fs-default}.

\subsubsection{\FSC}
The VRML file is next\_family.wrl.\newline
The results are similar to section~\ref{sec:fs-default}.

\subsubsection{\FSD}
The VRML file is serif\_family.wrl.\newline
The results are similar to section~\ref{sec:fs-default}.

\subsubsection{\FSE}
The VRML file is style.wrl.\newline
The results are similar to section~\ref{sec:fs-default}.

\subsubsection{\FSF}
The VRML file is size\_spacing.wrl.\newline
The sizes appear too large, and the size 2, spacing 0.5 combination overlaps.
The vertical text is rendered horizontally.  

\subsubsection{\FSG}
The VRML file is driver.wrl.\newline
FreeWRL crashes as both plug-in and stand-alone application:
\begin{verbatim}
FreeWRL NOTE: run with -best mode to get smooth shading
PARSE ERROR: ' horizontal "TRUE"
	
	
	directOutput TRUE
	url' XXX ' "javascript: 
	
	
	
	function getFile (value,',
in improper SFString because  at
/usr/lib/perl5/site_perl/5.6.0/i386-linux/VRML/Parser.pm line 66.
	VRML::Error::parsefail('^J^J^J^J^J^J^J^J^J^J^J^J^J^J^J^J^
J^J^J^J^J^J^M^JViewpoint {^M^J^Iposition 0 0 22^M^J}^M^J^M^J^M^JDE...',
'improper SFString')
called at /usr/lib/perl5/site_perl/5.6.0/i386-linux/VRML/VRMLFields.pm
line 752
	VRML::Field::SFString::parse('VRML::Field::SFString',
'VRML::Scene=HASH(0x81958dc)', '^J^J^J^J^J^J^J^J^J^J^J^J^J^J^J^J^J^J^J^J^
J^J^M^JViewpoint {^M^J^Iposition 0 0 22^M^J}^M^J^M^J^M^JDE...', undef)
called at /usr/lib/perl5/site_perl/5.6.0/i386-linux/VRML/VRMLFields.pm
line 1006
	VRML::Field::Multi::parse('VRML::Field::MFString',
'VRML::Scene=HASH(0x81958dc)',
'^J^J^J^J^J^J^J^J^J^J^J^J^J^J^J^J^J^J^J^J^J^J^M^JViewpoint
{^M^J^Iposition 0 0 22^M^J}^M^J^M^J^M^JDE...')
called at /usr/lib/perl5/site_perl/5.6.0/i386-linux/VRML/Parser.pm
line 229
	VRML::Parser::parse_interfacedecl('VRML::Scene=HASH(0x81958dc)',
0, 1, '^J^J^J^J^J^J^J^J^J^J^J^J^J^J^J^J^J^J^J^J^J^J^M^JViewpoint
{^M^J^Iposition 0 0 22^M^J}^M^J^M^J^M^JDE...', 1, '{', '}')
called at /usr/lib/perl5/site_perl/5.6.0/i386-linux/VRML/Parser.pm
line 262
	VRML::Parser::parse_script('VRML::Scene=HASH(0x81958dc)',
'^J^J^J^J^J^J^J^J^J^J^J^J^J^J^J^J^J^J^J^J^J^J^M^JViewpoint
{^M^J^Iposition 0 0 22^M^J}^M^J^M^J^M^JDE...')
called at /usr/lib/perl5/site_perl/5.6.0/i386-linux/VRML/Parser.pm
line 333
	VRML::Field::SFNode::parse('VRML::Field::SFNode',
'VRML::Scene=HASH(0x81958dc)',
'^J^J^J^J^J^J^J^J^J^J^J^J^J^J^J^J^J^J^J^J^J^J^M^JViewpoint
{^M^J^Iposition 0 0 22^M^J}^M^J^M^J^M^JDE...')
called at /usr/lib/perl5/site_perl/5.6.0/i386-linux/VRML/Parser.pm
line 307
	VRML::Field::SFNode::parse('VRML::Field::SFNode',
'VRML::Scene=HASH(0x81958dc)',
'^J^J^J^J^J^J^J^J^J^J^J^J^J^J^J^J^J^J^J^J^J^J^M^JViewpoint
{^M^J^Iposition 0 0 22^M^J}^M^J^M^J^M^JDE...')
called at /usr/lib/perl5/site_perl/5.6.0/i386-linux/VRML/Parser.pm
line 141
	VRML::Parser::parse_statement('VRML::Scene=HASH(0x81958dc)',
'^J^J^J^J^J^J^J^J^J^J^J^J^J^J^J^J^J^J^J^J^J^J^M^JViewpoint
{^M^J^Iposition 0 0 22^M^J}^M^J^M^J^M^JDE...')
called at /usr/lib/perl5/site_perl/5.6.0/i386-linux/VRML/Parser.pm
line 103
	VRML::Parser::parse('VRML::Scene=HASH(0x81958dc)',
'#VRML V2.0 utf8^M^J#  ^M^J# Module:  Appearance^M^J# Node  :
 FontStyl...')
called at /usr/lib/perl5/site_perl/5.6.0/i386-linux/VRML/Browser.pm
line 113
	VRML::Browser::load_string('VRML::Browser=HASH(0x861f088)',
'#VRML V2.0 utf8^M^J#  ^M^J# Module:  Appearance^M^J# Node  : 
FontStyl...',
'driver.wrl')
called at /usr/lib/perl5/site_perl/5.6.0/i386-linux/VRML/Browser.pm
line 105
	VRML::Browser::load_file('VRML::Browser=HASH(0x861f088)',
'driver.wrl', undef)
called at /usr/bin/freewrl line 541

Died at /usr/bin/freewrl line 392.

\end{verbatim}

\subsection{ImageTexture Test Cases}

\subsubsection{\ITA}
\label{sec:jpeg-rgb}
The VRML file is all\_jpg.wrl.

\subsubsection{\ITB}
The VRML file is all\_png.wrl.

\subsubsection{\ITC}
The VRML file is greyscale\_jpg.wrl.

\subsubsection{\ITD}
The VRML file is greyscale\_png.wrl.

\subsubsection{\ITE}
The VRML file is 256jpg.wrl.

\subsubsection{\ITF}
The VRML file is 256png.wrl.

\subsubsection{\ITG}
\label{sec:jpeg-ifs}
The VRML file is IndexedFaceSet.wrl.\newline
The ImageTexture is not mapped over the entire IndexedFaceSet, rather
it appears to be mapped to sections of the geometry.

\subsubsection{\ITHa\ITHb}
The VRML file is 3sides.wrl.\newline
The results are similar to section~\ref{sec:jpeg-ifs}.

\setcounter{subsubsection}{22}
%% The next subsubsection should be test case 23:
\subsubsection{\ITWa\ITWb}
The VRML file is rgb\_alpha\_png.wrl, and the ImageTexture is 8bcolrtp.png.\newline
The colours look a little washed-out.

\setcounter{subsubsection}{29}
%% The next subsubsection should be test case 30:
\subsubsection{\ITAD}
\label{sec:vert-clamp}
The VRML file is repeats.wrl, and the ImageTexture is vts.jpg.\newline
The texture appears to be clamped correctly, however the remaining portions are rendered
darker than expected.

\subsubsection{\ITAE}
The VRML file is repeatt.wrl, and the ImageTexture is vts.jpg.\newline
The results are similar to section~\ref{sec:vert-clamp}.

\setcounter{subsubsection}{32}
%% The next subsubsection should be test case 33:
\subsubsection{\ITAG}
The VRML file is repeats\_png.wrl, and the ImageTexture is vts.png.\newline
The results are similar to section~\ref{sec:vert-clamp}.

\subsubsection{\ITAH}
The VRML file is repeatt\_png.wrl, and the ImageTexture is vts.png.\newline
The results are similar to section~\ref{sec:vert-clamp}.

\subsection{Material Test Cases}

\subsubsection{\MatA}
The VRML file is default\_primitives.wrl.

\subsubsection{\MatB}
The VRML file is default\_complex.wrl.

\subsubsection{\MatC}
The VRML file is diffuseColor.wrl.\newline
The background, which should be dark grey not black, is too dark to see the black geometries.

\subsubsection{\MatD}
The VRML file is shininess.wrl.

\subsubsection{\MatE}
The VRML file is diffusecolor\_complex.wrl.\newline
The background, which should be dark grey not black, is too dark to see the black geometries.

\subsubsection{\MatF}
The VRML file is shininess\_complex.wrl.

\subsubsection{\MatGa \MatGb}
\label{sec:pref-colour}
The VRML file is texture\_first\_primitives.wrl.\newline
The texture colour does not supersede the blue diffuseColor field defined in the Material
node for all four primitive types. 

\subsubsection{\MatHa \MatHb}
The VRML file is texture\_first\_complex.wrl.\newline
The results are similar to section~\ref{sec:pref-colour}.

\subsubsection{\MatIa \MatIb}
The VRML file is texture\_over\_color\_complex.wrl.\newline
The results are similar to section~\ref{sec:pref-colour}.

\subsubsection{\MatJa \MatJb}
The VRML file is allgeoms\_diffusecolor\_texture.wrl.\newline
Improper texture mapping to the extrusion geometry.

\subsubsection{\MatKa \MatKb}
\subsubsection{\MatL}
The VRML file is complex\_color\_first.wrl.

\subsubsection{\MatMa \MatMb}
The VRML file is allgeoms\_diffusecolor.wrl.

\subsubsection{\MatNa \MatNb}
The VRML file is pointset\_color\_first.wrl.\newline
FreeWRL crashes as both plug-in and stand-alone application:
\begin{verbatim}
Not same number of colors and points at 
/usr/lib/perl5/site_perl/i386-linux/VRML/GLBackEnd.pm line 721.
    VRML::GLBackEnd::render('VRML::GLBackEnd=HASH(0x815a810)') called at 
/usr/lib/perl5/site_perl/i386-linux/VRML/GLBackEnd.pm line 226
    VRML::GLBackEnd::update_scene('VRML::GLBackEnd=HASH(0x815a810)',
 989940484.62) called at
/usr/lib/perl5/site_perl/i386-linux/VRML/Browser.pm line 204
    VRML::Browser::tick('VRML::Browser=HASH(0x86bb614)') called at 
/usr/lib/perl5/site_perl/i386-linux/VRML/Browser.pm line 127
    VRML::Browser::eventloop('VRML::Browser=HASH(0x86bb614)') called at 
/usr/bin/freewrl line 551

Died at /usr/bin/freewrl line 391.
\end{verbatim}

\subsubsection{\MatOa \MatOb}
The VRML file is pointset.wrl.\newline
The PointSet is not visible, rather the entire FreeWRL display is red in
both the plug-in and the stand-alone application.

\subsubsection{\MatPa \MatPb}
The VRML file is indexedlineset\_color\_first.wrl.

\subsubsection{\MatQa \MatQb}
The VRML file is indexedlineset.wrl.

\subsubsection{\MatRa \MatRb}
\subsubsection{\MatSa \MatSb}
\subsubsection{\MatTa \MatTb}
\subsubsection{\MatUa \MatUb}
\subsubsection{\MatV}
\subsubsection{\MatW}
\subsubsection{\MatX}
\subsubsection{\MatY}
\subsubsection{\MatZ}
\subsubsection{\MatAA}
\subsubsection{\MatAB}
\subsubsection{\MatAC}

\subsection{MovieTexture Test Cases}
MovieTexture is not supported. See bug 435228.

\subsection{PixelTexture Test Cases}

\subsection{TextureTransform Test Cases}

\subsubsection{\TTA}
\subsubsection{\TTBa \TTBb}
\subsubsection{\TTCa \TTCb}
\subsubsection{\TTDa \TTDb}
\subsubsection{\TTEa \TTEb}
\subsubsection{\TTFa \TTFb}
\subsubsection{\TTGa \TTGb}
\subsubsection{\TTHa \TTHb}

\section{Bindable Nodes Test Cases}
\resetTestCase

\begin{center}
\setlongtables
\begin{longtable}{|l|l|l|}
%% Table head and foot rows
\hline
\textbf{Bindable Node} & \textbf{Test Case} & \textbf{Result} \\
\hline\hline
\endhead
%%\cline{2-3}
 & \multicolumn{2}{|r|}{\textsl{continued on the next page\ldots}} \\
\hline
\endfoot
\hline
\endlastfoot
Background & \testCase \BkgA & Failed \\\cline{2-3}
 & \testCase \BkgB & Failed \\\cline{2-3}
 & \testCase \BkgCa & Failed \\
 & \BkgCb & \\\cline{2-3}
 & \testCase \BkgDa & Failed \\
 & \BkgDb & \\\cline{2-3}
 & \testCase \BkgE & Failed \\\cline{2-3}
 & \testCase \BkgFa & Passed \\
 & \BkgFb & \\\cline{2-3}
 & \testCase \BkgGa & \\
 & \BkgGb & \\\cline{2-3}
 & & \\\hline
\resetTestCase
Fog & \testCase \FogA & Passed \\\cline{2-3}
 & \multicolumn{2}{|c|}{\textbf{Fog not supported.}} \\\hline
\resetTestCase
NavigationInfo & & \\\cline{2-3}
 & & \\\hline
\resetTestCase
Viewpoint & & \\\cline{2-3}
 & & \\
\end{longtable}
\end{center}

\subsection{Background Test Cases}
\subsection{Fog Test Cases}
Fog is not supported. See bug 435229.

\subsubsection{\FogA}
The VRML file is nofog.wrl.

\subsection{NavigationInfo Test Cases}
\section{Geometric Properties Test Cases}
\resetTestCase

\begin{center}
\setlongtables
\begin{longtable}{|l|l|l|}
%% Table head and foot rows
\hline
\textbf{Node Group} & \textbf{Test Case} & \textbf{Result} \\
\hline\hline
\endhead
%%\cline{2-3}
 & \multicolumn{2}{|r|}{\textsl{continued on the next page\ldots}} \\
\hline
\endfoot
\hline
\endlastfoot
& & \\
\cline{2-3}
\end{longtable}
\end{center}

\section{Interpolators Test Cases}
\resetTestCase

\begin{center}
\setlongtables
\begin{longtable}{|l|l|l|}
%% Table head and foot rows
\hline
\textbf{Node Group} & \textbf{Test Case} & \textbf{Result} \\
\hline\hline
\endhead
%%\cline{2-3}
 & \multicolumn{2}{|r|}{\textsl{continued on the next page\ldots}} \\
\hline
\endfoot
\hline
\endlastfoot
& & \\
\cline{2-3}
\end{longtable}
\end{center}

\section{Lights Test Cases}
\resetTestCase

\begin{center}
\setlongtables
\begin{longtable}{|l|l|l|}
%% Table head and foot rows
\hline
\textbf{Node Group} & \textbf{Test Case} & \textbf{Result} \\
\hline\hline
\endhead
%%\cline{2-3}
 & \multicolumn{2}{|r|}{\textsl{continued on the next page\ldots}} \\
\hline
\endfoot
\hline
\endlastfoot
& & \\
\cline{2-3}
\end{longtable}
\end{center}

\section{Miscellaneous Test Cases}
\resetTestCase

\begin{center}
\setlongtables
\begin{longtable}{|l|l|l|}
%% Table head and foot rows
\hline
\textbf{Node Group} & \textbf{Test Case} & \textbf{Result} \\
\hline\hline
\endhead
%%\cline{2-3}
 & \multicolumn{2}{|r|}{\textsl{continued on the next page\ldots}} \\
\hline
\endfoot
\hline
\endlastfoot
& & \\
\cline{2-3}
\end{longtable}
\end{center}

\section{Sensors Test Cases}
\resetTestCase

\begin{center}
\setlongtables
\begin{longtable}{|l|l|l|}
%% Table head and foot rows
\hline
\textbf{Node Group} & \textbf{Test Case} & \textbf{Result} \\
\hline\hline
\endhead
%%\cline{2-3}
 & \multicolumn{2}{|r|}{\textsl{continued on the next page\ldots}} \\
\hline
\endfoot
\hline
\endlastfoot
& & \\
\cline{2-3}
\end{longtable}
\end{center}

\section{Sounds Test Cases}
\resetTestCase

\begin{center}
\setlongtables
\begin{longtable}{|l|l|l|}
%% Table head and foot rows
\hline
\textbf{Node Group} & \textbf{Test Case} & \textbf{Result} \\
\hline\hline
\endhead
%%\cline{2-3}
 & \multicolumn{2}{|r|}{\textsl{continued on the next page\ldots}} \\
\hline
\endfoot
\hline
\endlastfoot
& & \\
\cline{2-3}
\end{longtable}
\end{center}

\subsubsection{}
The VRML file is 

\section{Special Groups Test Cases}
\resetTestCase

\begin{center}
\setlongtables
\begin{longtable}{|l|l|l|}
%% Table head and foot rows
\hline
\textbf{Special Groups} & \textbf{Test Case} & \textbf{Result} \\
\hline\hline
\endhead
%%\cline{2-3}
 & \multicolumn{2}{|r|}{\textsl{continued on the next page\ldots}} \\
\hline
\endfoot
\hline
\endlastfoot
Inline & \testCase \SpGrA & Passed \\\cline{2-3}
& \testCase \SpGrB & Passed \\\cline{2-3}
& \testCase \SpGrC & Passed \\\cline{2-3}
& \testCase \SpGrD & Passed \\\cline{2-3}
& \testCase \SpGrE & Passed \\\hline
\multicolumn{3}{|c|}{\textbf{Beyond minimum conformance requirements:
bounding boxes.}} \\\hline
& \testCase \SpGrF & Failed \\\hline
\resetTestCase
LOD & & \\\cline{2-3}
& & \\\hline
\resetTestCase
Switch & & \\\cline{2-3}
& & \\
\end{longtable}
\end{center}

\subsection{Inline Test Cases}

\subsubsection{\SpGrA}
The VRML file is single-url.wrl.

\subsubsection{\SpGrB}
The VRML file is nested-inline.wrl.

\subsubsection{\SpGrC}
The VRML file is invalid-url.wrl.
See bug 435578.

\subsubsection{\SpGrD}
The VRML file is absolute-url.wrl.

\subsubsection{\SpGrE}
The VRML file is relative-url.wrl.

\subsubsection{\SpGrF}
The VRML file is test-bboxsizesame.wrl.\newline
The ImageTexture from the Inline'd file ref1.wrl does not get
applied to the IndexedFaceSet geometry.

\end{document}
